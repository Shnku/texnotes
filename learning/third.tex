\documentclass{article}
\usepackage{amsmath}
\usepackage{physics}
\usepackage{siunitx}
\usepackage{graphicx}
\usepackage{multicol}
\usepackage[left=1.50cm, right=2.50cm, top=2.00cm, bottom=1.00cm]{geometry}
\setlength{\columnsep}{2cm}
\begin{document}

\begin{multicols}{2}
	\section{Introduction}

	This is an introductionhjgda section with some bullet points:

	\begin{itemize}
		\item This is the \large\textbf{firxzxzst} bullet point.
		\item This is the second bullet point.
		\item This is the third bullet point.
	\end{itemize}
	Now, here are some numbered points:
	\begin{enumerate}
		\item This is the first numbered point ok.
		\item This is the seconeed numbered point.
		\item This is the third numbered point.
	\end{enumerate}
	You can also nest the environments to create more complex lists:
	\begin{enumerate}
		\item This is the first numbered point.
		      \begin{itemize}
			      \item This is a bullet point under the first numbered point.
			      \item This is another bullet point under the first numbered point.
		      \end{itemize}
		\item This is the second numbered point.
		      \begin{enumerate}
			      \item This is the first sub-numbered point under the second numbered point.
			      \item This is the second sub-numbered point under the second numbered point.
		      \end{enumerate}
		\item This is the third numbereahdad point.
	\end{enumerate}
	% Your content goes here
	\section{Introduction}
	This is the introduction section.

	\section{Equations}
	Here are some equations:

	\begin{equation}
		E = mc^2
	\end{equation}
	\begin{align}
		F & = ma          \\
		v & = \frac{d}{t}
	\end{align}

	\section{Figures}
	\begin{figure}[h]
		\centering
		%\includegics[width=0.8\linewidth]{example-image.png}
		\caption{An example figure.}
		\label{fig:example}
	\end{figure}

\end{multicols}

\end{document}
\documentclass{article}
\usepackage{amsmath}
\usepackage{physics}
\usepackage{siunitx}
\usepackage{graphicx}
\usepackage{multicol}

\begin{document}

\begin{multicols}{2}

	% Your content goes here
	\section{Introduction}
	This is the introduction section.

	\section{Equations}
	Here are some equations:

	\begin{equation}
		E = mc^2
	\end{equation}

	\begin{align}
		F & = ma          \\
		v & = \frac{d}{t}
	\end{align}

	\section{Figures}
	\begin{figure}[h]
		\centering
		\includegics[width=0.8\linewidth]{example-image.png}
		\caption{An example figure.}
		\label{fig:example}
	\end{figure}

\end{multicols}

\end{document}
